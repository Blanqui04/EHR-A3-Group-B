\section{Discussion}

This retrospective study examined the epidemiology and in-hospital mortality of sepsis among ICU admissions using the MIMIC-III database. The key findings indicate a substantial clinical burden of sepsis and meaningful associations with mortality outcomes.

\subsection{Sepsis Prevalence and Epidemiology}

The study identified sepsis in 13.59\% of ICU admissions (7,261 cases). This prevalence aligns with previously reported estimates in the literature, though variations across studies reflect differences in sepsis definitions, patient populations, and case identification methods. Previous ICU-based epidemiological research has reported sepsis occurrence rates in the range of approximately 13\% to 40\% among critically ill patients, reflecting heterogeneity in populations and case definitions \cite{icu_sepsis}. The ICD-9-CM-based approach used in this analysis captures clinically documented sepsis diagnoses and represents a practical definition consistent with retrospective database studies. A systematic review of ICU-treated sepsis studies has shown that prevalence measures vary widely but can include proportions comparable to our findings \cite{epidemiology_icu}.

\subsection{Impact of Sepsis on In-Hospital Mortality}

A striking finding was the substantial difference in in-hospital mortality between sepsis and non-sepsis patients (32.57\% vs. 9.05\%). This 3.6-fold elevation in mortality among sepsis patients underscores the severe clinical impact of sepsis in the ICU setting and is consistent with prior literature demonstrating sepsis as a major driver of critical illness and death \cite{icu_sepsis}. The absolute risk difference of approximately 23 percentage points indicates that sepsis contributes substantially to in-hospital mortality in ICU patients.

\subsection{Predictive Model Development and Performance}

The logistic regression model demonstrated modest discrimination ability, with an AUC of 0.622 on the test set. This limited discrimination indicates that demographic and administrative variables alone (age, gender, ethnicity, insurance, and ICD-9 code) have restricted predictive power for in-hospital mortality in sepsis patients. The optimal classification threshold of 0.334 (rather than the conventional 0.5) reflects the clinical context: in a high-mortality population, using a lower threshold appropriately balances sensitivity and specificity.

The model achieved a sensitivity of 60.3\%, capturing a substantial proportion of patients who will die in-hospital. However, the specificity of 57.3\% indicates moderate ability to correctly identify survivors, resulting in a notable false positive rate. The positive predictive value (40.6\%) reflects the challenge of mortality prediction: when the model predicts mortality, it is correct less than half the time. The negative predictive value (74.9\%) provides reasonable confidence when predicting survival. These performance characteristics suggest that the model has limited clinical utility for individual risk stratification but confirms that demographic factors alone are insufficient for accurate mortality prediction.

\subsection{Interpretation of model performance}

The modest discriminative performance of the logistic regression model (AUC of 0.622) highlights a fundamental limitation of the approach: demographic and administrative variables do not capture the physiologic complexity of sepsis severity. Clinical variables such as vital signs, lactate levels, organ dysfunction scores (SOFA, APACHE), and infection source are likely necessary to achieve clinically meaningful discrimination \cite{seymour}.

\subsection{Clinical and Methodological Implications}

The model coefficients suggest that age is associated with increased mortality risk, consistent with extensive literature on aging and sepsis outcomes \cite{sepsis3}. Gender and ethnicity showed variable associations, though interpretation is limited by the modest overall model performance. Insurance status may serve as a proxy for socioeconomic factors and access to care.

The limited predictive performance (AUC = 0.622) carries an important clinical implication: demographic factors alone should not be used for mortality risk stratification in sepsis patients. Effective risk assessment requires integration of clinical parameters including physiologic measurements, laboratory values, and validated severity scores.

\subsection{Limitations}

Several limitations warrant acknowledgment. First, the use of ICD-9-CM codes depends on clinical documentation and coding practices, which may introduce misclassification, with sepsis potentially underdiagnosed or overdiagnosed in some cases. Second, the model includes only demographic and administrative variables; clinical measurements such as vital signs, laboratory values, and organ dysfunction markers were not included and could improve discriminatory performance. Third, MIMIC-III represents a single tertiary academic medical center in the United States, limiting generalizability to other healthcare settings or populations. Fourth, the observational nature of the study precludes causal inference regarding risk factors and outcomes. Fifth, the inclusion of multiple ICU stays or admissions from the same patient may have introduced dependence between observations, potentially affecting the interpretation of the results. Sixth, sepsis classification was performed at the admission level and may not fully capture temporal changes in sepsis status during the ICU stay. Finally, age values were used as provided in the MIMIC-III database, where ages above 89 years are intentionally obfuscated for privacy reasons. This may have introduced imprecision in age-related analyses, particularly among the oldest patients.

\subsection{Future Directions}

Future work could enhance the model by incorporating physiologic and laboratory variables available in MIMIC-III, such as vital signs, lactate, organ dysfunction scores, and infection source. Validation in external cohorts, including other ICU populations or more recent data, would strengthen confidence in model generalizability. Investigation of temporal trends in sepsis epidemiology and outcomes could provide insight into changes in awareness and management strategies over time. Finally, comparative effectiveness research evaluating specific clinical interventions in sepsis management could help inform practice guidelines and improve patient outcomes.
