\begin{abstract}
\noindent\textbf{Scientific abstract.}  
Sepsis is a life-threatening organ dysfunction caused by a dysregulated host response to infection and remains a leading cause of mortality in intensive care units (ICUs). Using the Medical Information Mart for Intensive Care III (MIMIC-III) database, we conducted a retrospective cohort study to estimate the proportion of adult ICU admissions with a sepsis diagnosis (defined by ICD-9-CM codes) and to compare in-hospital mortality between sepsis and non-sepsis admissions. Our cohort included all adult patients (age $\geq$16 years) admitted to the ICU between 2001 and 2012. This report details the cohort definition and methodology for subsequent epidemiological analysis.
\end{abstract}

\vspace{0.5cm}

\noindent\textbf{Non-scientific abstract.}  
Imagine your body is a city, and an infection is like a fire. Normally, firefighters (your immune system) put it out quickly. But in sepsis, the firefighters go crazy—they start breaking buildings trying to stop the fire, hurting the city itself. We looked at thousands of ICU patients in a public database to find out how many had this “fire emergency” (sepsis) and whether they were more likely to die in the hospital than other patients. First, we had to define exactly who was in our study—only adults admitted to the ICU—and that’s what this part explains.