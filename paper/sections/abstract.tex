\begin{abstract}
\noindent\textbf{Scientific Abstract}
Sepsis is a life-threatening condition caused by a dysregulated host response to infection, representing a major cause of morbidity and mortality in intensive care units (ICUs). Understanding its epidemiology is essential for improving patient outcomes and resource allocation.
We conducted a retrospective cohort study using the MIMIC-III database, analyzing 53,423 adult ICU admissions from Beth Israel Deaconess Medical Center (2001--2012). Sepsis cases were identified using ICD-9-CM codes (995.91, 995.92, 785.52). We estimated sepsis prevalence, compared in-hospital mortality between sepsis and non-sepsis patients, and developed a logistic regression model to predict mortality among sepsis patients.
Sepsis was present in 7,261 ICU admissions (13.59\%). In-hospital mortality was significantly higher among sepsis patients (32.57\%) compared to non-sepsis patients (9.05\%). The logistic regression model achieved an AUC of 0.622, with age and male sex identified as significant predictors of mortality.
Sepsis affects a substantial proportion of ICU patients and is associated with markedly elevated mortality. These findings underscore the importance of early recognition and intervention strategies in critical care settings.

\bigskip
\noindent\textbf{None Scientific Abstract}
Imagine you're watching a hospital like it's a video game: patients come in, get treated, and (hopefully) go home feeling better. But sometimes, something dangerous called \textit{sepsis} sneaks in, it's when the body overreacts to an infection and starts hurting itself. Sepsis can be really serious, especially in the ICU.
In our project, we used a huge database of real (but anonymous) hospital records to find out how often sepsis happens and how likely it is to lead to death during the hospital stay. We taught a computer to spot sepsis by reading diagnosis codes, like digital detective work, and then compared patients with and without sepsis. We found that people with sepsis were much more likely to die in the hospital, even after accounting for age and sex.
This isn't just about numbers, it shows how we can use past health data to understand hidden risks and maybe even build warning systems to help doctors act faster in the future. Think of it as using data to give doctors super-powered hindsight,  so they can protect more lives tomorrow.
\end{abstract}



