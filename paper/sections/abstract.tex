\begin{abstract}
Imagine you’re watching a hospital like it’s a video game: patients come in, get treated, and (hopefully) go home feeling better. But sometimes, something dangerous called *sepsis* sneaks in—it’s when the body overreacts to an infection and starts hurting itself. Sepsis can be really serious, especially in the ICU.

In our project, we used a huge database of real (but anonymous) hospital records to find out how often sepsis happens and how likely it is to lead to death during the hospital stay. We taught a computer to spot sepsis by reading diagnosis codes—like digital detective work—and then compared patients with and without sepsis. We found that people with sepsis were much more likely to die in the hospital, even after accounting for age and sex.

This isn’t just about numbers—it shows how we can use past health data to understand hidden risks and maybe even build warning systems to help doctors act faster in the future. Think of it as using data to give doctors super-powered hindsight… so they can protect more lives tomorrow.
\end{abstract}
