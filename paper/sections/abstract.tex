\begin{abstract}
This study focuses on sepsis, a life-threatening condition caused by the body's extreme response to infection. Sepsis is one of the leading causes of death in hospitals, especially in Intensive Care Units (ICUs) where the sickest patients are treated.
In this study our aim was to analyze medical records from over 60,000 ICU patients to understand how common sepsis is and who is most at risk of dying from it. We used a large database called MIMIC-III, which contains anonymous health data from a hospital in Boston.
As a result of this study we fin that about 1 in 12 ICU patients (8.4\%) had sepsis. The really striking finding: patients with sepsis were \textbf{2.5 times more likely to die} in hospital compared to other ICU patients (32\% vs 13\%). We also tried to predict who would survive using basic information like age, gender, and insurance type---but this only worked about 62\% of the time, which isn't great. This tells us that to really predict who's at risk, doctors need more detailed medical information like blood tests and vital signs.
This result it's imporant as Sepsis kills more people than heart attacks or strokes, yet many people have never heard of it. Our study confirms that sepsis is extremely dangerous and that hospitals need better tools to identify high-risk patients early. The sooner doctors recognize sepsis, the better the chances of survival.
\end{abstract}

