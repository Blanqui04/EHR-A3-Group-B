\section{Conclusion}

This retrospective analysis of sepsis epidemiology and in-hospital mortality in ICU admissions demonstrates the substantial clinical burden of sepsis and identifies demographic factors associated with increased mortality risk. The study leveraged the MIMIC-III database to define sepsis cases using ICD-9-CM diagnostic codes and analyzed outcomes among 7,261 sepsis cases identified in a cohort of ICU admissions.

Sepsis prevalence of 13.59\% among ICU admissions and a 3.6-fold elevation in mortality among sepsis patients (32.57\% vs. 9.05\%) underscore the severe clinical impact of this syndrome. A logistic regression model developed to predict mortality among sepsis patients achieved an AUC of 0.622, indicating that demographic and administrative factors alone have limited discriminative ability for mortality prediction. The model satisfies standard statistical assumptions but demonstrates that clinical variables are necessary for meaningful risk stratification.

These findings reinforce the recognized importance of sepsis in critical care and provide evidence-based context for clinical decision-making in ICU settings. The high mortality associated with sepsis validates the urgent need for rapid recognition and aggressive management. The modest performance of the demographic-based predictive model indicates that effective risk stratification requires incorporation of clinical and physiologic variables including vital signs, lactate levels, and SOFA scores.

Future research incorporating physiologic and laboratory variables, validation in external cohorts, and evaluation of clinical interventions will be necessary to develop clinically meaningful prediction tools and improve outcomes for critically ill patients with sepsis.
