This retrospective analysis of sepsis epidemiology and in-hospital mortality in ICU admissions demonstrates the substantial clinical burden of sepsis and identifies demographic factors associated with increased mortality risk. The study leveraged the MIMIC-III database to define sepsis cases using ICD-9-CM diagnostic codes and analyzed outcomes among 5,149 sepsis cases identified in a cohort of ICU admissions.

\subsection{Key Findings}

Sepsis prevalence of 8.4\% among ICU admissions and a 2.5-fold elevation in mortality among sepsis patients (32.1\% vs. 12.8\%) underscore the severe clinical impact of this syndrome. A logistic regression model developed to predict mortality among sepsis patients achieved an AUC of 0.622, indicating that demographic and administrative factors alone have limited discriminative ability for mortality prediction. The model satisfies standard statistical assumptions (multicollinearity assessment) but demonstrates that clinical variables are necessary for meaningful risk stratification.

\subsection{Clinical Significance}

These findings reinforce the recognized importance of sepsis in critical care and provide evidence-based context for clinical decision-making in ICU settings. The high mortality associated with sepsis validates the urgent need for rapid recognition and aggressive management. The modest performance of the demographic-based predictive model (AUC = 0.622) indicates that effective risk stratification requires incorporation of clinical and physiologic variables beyond demographic factors.

\subsection{Methodological Contribution}

This study demonstrates the utility of administrative databases for retrospective sepsis epidemiology research. By defining cases using ICD-9 codes, the analysis reflects real-world clinical practice and is reproducible in other healthcare settings using standard administrative data. The consistent application of case definitions and systematic model validation (including cross-validation and overfitting assessment) strengthen confidence in the findings.

\subsection{Conclusion}

In conclusion, this study provides a characterization of sepsis epidemiology and outcomes in a large ICU population. Sepsis remains a important cause of in-hospital mortality, with a prevalence of 8.4\% and mortality rate of 32.1\% among identified cases. The logistic regression model achieved an AUC of 0.622, demonstrating that demographic factors alone are insufficient for accurate mortality prediction. Future research incorporating physiologic and laboratory variables (e.g., vital signs, lactate, SOFA scores), validation in external cohorts, and investigation of intervention effectiveness will be necessary to develop clinically useful risk stratification tools for sepsis patients in critical care.

% resum
In conclusion, this study provides a retrospective characterization of sepsis epidemiology and in-hospital mortality in a large ICU population. Sepsis represents an important cause of in-hospital mortality, with a prevalence of 8.4\% and a mortality rate of 32.1\% among identified cases.

A logistic regression model based on demographic and administrative variables achieved limited discrimination for mortality prediction (AUC = 0.622), highlighting that such variables alone are insufficient for accurate risk stratification in sepsis patients. These findings underscore the need for future research incorporating physiologic and laboratory variables (e.g., vital signs, lactate, SOFA scores), validation in external cohorts, and evaluation of clinical interventions to develop clinically meaningful prediction tools and improve outcomes for critically ill patients with sepsis.
