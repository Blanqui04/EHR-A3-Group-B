\subsection{Data source}
We used MIMIC-III (version 1.4), a publicly available, de-identified database containing comprehensive clinical data from over 40,000 ICU admissions at Beth Israel Deaconess Medical Center (Boston, USA) between 2001 and 2012 \cite{mimic}. The database includes demographics, vital signs, laboratory tests, medications, imaging reports, and ICD-9 diagnosis codes. Access was granted after completion of the required CITI training program.

\subsection{Cohort definition}
We performed a retrospective observational study using the Medical Information Mart for Intensive Care III (MIMIC-III), version 1.4, a publicly available database containing de-identified health data from over 40,000 adult and neonatal ICU admissions at Beth Israel Deaconess Medical Center between 2001 and 2012 \cite{mimic}. Our cohort comprised all adult patients aged 16 years or older with at least one ICU admission recorded in the \texttt{ICUSTAYS} table. Age was calculated as the difference in years between the ICU admission time (\texttt{INTIME}) and the patient’s date of birth. For patients with multiple ICU stays, each stay was considered an independent observation. Neonatal and pediatric admissions (age < 16 years) were excluded. The final analytical cohort included all qualifying ICU admissions available in the database, without further exclusion criteria.

All data extraction and cohort construction were performed using R (version 4.4.0) with the \texttt{DBI} and \texttt{dplyr} packages, connecting directly to a local PostgreSQL instance of MIMIC-III. The full reproducible script is available in the project repository (see \texttt{analysis/01\_cohort.R}).