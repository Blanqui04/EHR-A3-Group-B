The complete R code used in this analysis is available in the Jupyter Notebook \texttt{Activity-A3.ipynb} located in the \texttt{analysis/} directory of this project repository. Below we present the key code segments organized by analysis section.

\subsection*{A.1 Data Loading and Cohort Construction}

\begin{verbatim}
# Load required libraries
library(dplyr)
library(tidyr)
library(lubridate)
library(stringr)
library(RMariaDB)
library(DBI)
library(ggplot2)

# Connect to MIMIC-III database
con <- dbConnect(
  drv = RMariaDB::MariaDB(),
  host = "ehr3.deim.urv.cat",
  dbname = "mimiciiiv14",
  port = 3306
)

# Build cohort: join tables and filter adults (age >= 16)
cohort_model <- tbl(con, "ICUSTAYS") %>%
  inner_join(tbl(con, "ADMISSIONS"), 
             by = c("HADM_ID", "SUBJECT_ID")) %>%
  inner_join(tbl(con, "PATIENTS"), 
             by = "SUBJECT_ID") %>%
  inner_join(tbl(con, "DIAGNOSES_ICD"), 
             by = c("HADM_ID", "SUBJECT_ID")) %>%
  mutate(age = year(INTIME) - year(DOB)) %>%
  filter(age >= 16) %>%
  select(HADM_ID, SUBJECT_ID, ICUSTAY_ID, age, 
         GENDER, ETHNICITY, INSURANCE, 
         ICD9_CODE, HOSPITAL_EXPIRE_FLAG) %>%
  collect()
\end{verbatim}

\subsection*{A.2 Sepsis Identification}

Sepsis was identified using ICD-9 diagnosis codes: 038.x (septicemia), 99591 (sepsis), 99592 (severe sepsis), and 78552 (septic shock).

\begin{verbatim}
# Create binary sepsis indicator
cohort_final <- cohort_model %>%
  mutate(sepsis = if_else(
    str_starts(icd9_code, "038") | 
    icd9_code %in% c("99591", "99592", "78552"), 
    1, 0))

# Ensure one row per ICU stay, prioritizing sepsis diagnoses
cohort_unique <- cohort_final %>%
  group_by(subject_id, hadm_id, icustay_id) %>%
  arrange(desc(sepsis)) %>%
  slice(1) %>%
  ungroup()
\end{verbatim}

\subsection*{A.3 Prevalence and Mortality Analysis}

\begin{verbatim}
# Calculate sepsis prevalence
sepsis_prevalence <- cohort_unique %>%
  summarise(
    total_icu_admissions = n(),
    sepsis_icu_admissions = sum(sepsis == "1"),
    sepsis_prevalence = (sepsis_icu_admissions / 
                         total_icu_admissions) * 100
  )

# Calculate mortality rates by sepsis status
mortality_sepsis <- cohort_unique %>%
  filter(sepsis == 1) %>%
  group_by(hospital_expire_flag) %>%
  summarise(count = n()) %>%
  mutate(percentage = count / sum(count) * 100)
\end{verbatim}

\subsection*{A.4 Logistic Regression Model}

\begin{verbatim}
library(caTools)
library(pROC)
library(car)

# Filter sepsis patients for modeling
patients_sepsis <- cohort_unique %>% filter(sepsis == 1)

# Split data (90% train, 10% test)
set.seed(100)
spl <- sample.split(patients_sepsis$hospital_expire_flag, 
                    SplitRatio = 0.9)
train <- subset(patients_sepsis, spl == TRUE)
test <- subset(patients_sepsis, spl == FALSE)

# Fit logistic regression model
logistic <- glm(hospital_expire_flag ~ age + gender + 
                ethnicity + insurance + icd9_code, 
                data = train, family = 'binomial')

# Generate predictions and evaluate
predict_test <- predict(logistic, type = "response", 
                        newdata = test)
roc_curve <- roc(response = test$hospital_expire_flag, 
                 predictor = predict_test)
auc_value <- auc(roc_curve)

# Find optimal threshold
coords <- coords(roc_curve, "best", 
                 best.method = "closest.topleft")
best_threshold <- coords$threshold

# Calculate performance metrics
predicted_class <- ifelse(predict_test >= best_threshold, 1, 0)
confusion_matrix <- table(test$hospital_expire_flag, 
                          predicted_class)

TN <- confusion_matrix[1, 1]
FP <- confusion_matrix[1, 2]
FN <- confusion_matrix[2, 1]
TP <- confusion_matrix[2, 2]

Sensitivity <- TP / (TP + FN)
Specificity <- TN / (TN + FP)
PPV <- TP / (TP + FP)
NPV <- TN / (TN + FN)

# Check multicollinearity
vif(logistic)
\end{verbatim}

\subsection*{A.5 Software Environment}

All analyses were performed using R version 4.x with the following key packages:
\begin{itemize}
    \item \texttt{dplyr}, \texttt{tidyr}: Data manipulation
    \item \texttt{RMariaDB}, \texttt{DBI}: Database connectivity
    \item \texttt{ggplot2}: Data visualization
    \item \texttt{caTools}: Data splitting
    \item \texttt{pROC}: ROC curve analysis
    \item \texttt{car}: Variance inflation factor calculation
\end{itemize}

