\subsection{Sepsis Prevalence}

We analyzed a total of 53,417 ICU admissions, from which \textbf{7,261 admissions (13.59\%)} were associated with a sepsis diagnosis based on ICD-9-CM codes. The sepsis definition encompassed patients with any of the following codes: septicemia (038.x), sepsis (995.91), severe sepsis (995.92), or septic shock (785.52).

As shown in Figure~\ref{fig:sepsis_distribution}, the majority of ICU admissions were not associated with sepsis (46,156 admissions, 86.41\%), nevertheless admissions related to sepsis accounted approximately for one in seven in the cohort. 

\begin{figure}[htbp]
\centering
\includegraphics[width=\columnwidth]{figures/sepsis_distribution_icu.pdf}
\caption{Distribution of Sepsis Among ICU Admissions. Of the total ICU admissions analyzed, 7,261 (13.59\%) were classified as sepsis cases based on ICD-9-CM diagnostic codes.}
\label{fig:sepsis_distribution}
\end{figure}

\subsection{Sepsis Prevalence by Demographics}

To explore demographic patterns, sepsis prevalence was stratified by gender (Figure~\ref{fig:sepsis_gender}) and age group (Figure~\ref{fig:sepsis_age}).

Sepsis prevalence was similar between males and females: 3,179 of 23,323 admissions among females (13.63\%) and 4,082 of 30,094 admissions among males (13.56\%), as seen in Figure~\ref{fig:sepsis_gender}. Taking a look into the age, sepsis prevalence increased across age groups, from 8.4\% in patients aged from 16 to 40 years, to 16\%  in patients aged 80 years and older, as seen in Figure~\ref{fig:sepsis_age}.

\begin{figure}[htbp]
\centering
\includegraphics[width=\columnwidth]{figures/sepsis_prevelence_by_gender.pdf}
\caption{Sepsis prevalence by gender.}
\label{fig:sepsis_gender}
\end{figure}

\begin{figure}[htbp]
\centering
\includegraphics[width=\columnwidth]{figures/sepsis_prevelence_by_age.pdf}
\caption{Sepsis prevalence stratified by age group (age groups being 16-40, 40-60, 60-80, 80+).}
\label{fig:sepsis_age}
\end{figure}

\subsection{Mortality Stratification by Sepsis Status}

In-hospital mortality among ICU admissions differed between the patients diagnosed with sepsis (Figures~\ref{fig:mortality_sepsis}) and the patients without a sepsis diagnosis (\ref{fig:mortality_non_sepsis}). Among ICU admissions classified as sepsis, the mortality rate was \textbf{32.57\%}, compared with \textbf{9.05\%} among patients without a sepsis diagnosis. This represents an approximately 3.6-fold increase in mortality proportion in the sepsis group relative to the non-sepsis group.

\begin{figure}[htbp]
\centering
\includegraphics[width=\columnwidth]{figures/mortality_of_patients_by_sepsis.pdf}
\caption{In-hospital mortality distribution among ICU admissions \textbf{with} sepsis. Approximately one-third of sepsis patients (32.57\%) died during hospitalization.}
\label{fig:mortality_sepsis}
\end{figure}

\begin{figure}[htbp]
\centering
\includegraphics[width=\columnwidth]{figures/mortality_of_patients_without_sepsis.pdf}
\caption{In-hospital mortality distribution among ICU admissions \textbf{without} sepsis. Mortality rate was 9.05\%.}
\label{fig:mortality_non_sepsis}
\end{figure}


\newpage
\subsection{Logistic Regression Model}

A logistic regression model was developed to predict in-hospital mortality among sepsis patients using the following predictor variables:

\begin{itemize}
    \item Age 
    \item Gender 
    \item Ethnicity 
    \item Insurance 
    \item ICD-9 Code 
\end{itemize}

The model was trained on a randomly selected 90\% subset of the sepsis cohort (using \texttt{set.seed(100)} for reproducibility) and tested on the remaining 10\%.

In the fitted model, age was associated with in-hospital mortality ($\beta = 0.00123$, $p = 0.00565$). Several ICD-9-CM subcategories were also associated with mortality, including the codes: 038.3 ($\beta = 0.691$, $p = 0.00394$), 038.8 ($\beta = 0.515$, $p = 0.01048$), 038.9 ($\beta = 0.660$, $p < 0.001$), 785.52 ($\beta = 1.073$, $p < 0.001$), and 995.92 ($\beta = 1.032$, $p < 0.001$). Gender, insurance categories, and ethnicity categories were not statistically significant at the 0.05 level in this model.


\subsection{Model Diagnostics and Assumptions}

\subsubsection{Multicollinearity Assessment (VIF)}

Variance Inflation Factors (VIF) were calculated to assess multicollinearity among the five predictors in the logistic regression model. VIF quantifies how much the variance of a regression coefficient is inflated due to multicollinearity with other predictors. We interpret VIF values as follows: VIF $< 5$ indicates no multicollinearity concern, VIF $5$--$10$ suggests moderate multicollinearity, and VIF $> 10$ indicates severe multicollinearity.

VIF was computed for each predictor from the fitted logistic regression model using auxiliary linear regressions. The aggregated results across the five main predictors are presented in Table~\ref{tab:vif}.

\begin{table}[htbp]
\centering
\caption{Variance Inflation Factors (VIF) for all model predictors from the logistic regression model (aggregated values).}
\label{tab:vif}
\begin{tabular}{lc}
\toprule
\textbf{Predictor} & \textbf{VIF} \\
\midrule
Age & 1.13 \\
Gender & 1.04 \\
ICD-9 Code & 1.67 \\
Insurance & 7.89 \\
Ethnicity & 43.70 \\
\bottomrule
\end{tabular}
\end{table}

Age (VIF $= 1.13$), Gender (VIF $= 1.04$), and ICD-9 Code (VIF $= 1.67$) show minimal inflation, indicating no multicollinearity concern for these predictors. Insurance (VIF $= 7.89$) exhibits moderate multicollinearity, while Ethnicity (VIF $= 43.70$) exhibits severe multicollinearity. However, the elevated VIF for categorical variables with many categories primarily reflects the inherent redundancy in dummy variable encoding rather than problematic multicollinearity between distinct predictors. The high VIF for Ethnicity is expected given the large number of ethnic categories in the MIMIC-III database. While the model exhibits some multicollinearity in these categorical predictors, this does not substantially compromise the interpretability of the coefficients for the continuous and binary predictors (Age, Gender, and ICD-9 Code).
\subsubsection{Independence of Observations}
Logistic regression assumes independence of observations. However, this assumption may be partially violated in our dataset: the same patient (\texttt{subject\_id}) may contribute multiple hospital admissions (\texttt{hadm\_id}), and the same admission may include multiple ICU stays (\texttt{icustay\_id}). This potential correlation between observations is acknowledged as a limitation of the analysis.

\subsection{Test Set Performance}

\subsubsection{Discrimination: ROC Curve and AUC}

We evaluated the model discrimination performance using a ROC analysis. The ROC curve on the test obtained an Area Under the Curve (AUC) of \textbf{0.622}, indicating modest discriminative ability (Figure~\ref{fig:roc_curve}). Using the closest-to-top-left criterion, maximizing sensitivity and specificity, the optimal probability threshold was determined to be \textbf{0.334}. At this threshold, the resulting confusion matrix yielded an overall classification accuracy of \textbf{58.3\%} on the test set. 

\begin{figure}[htbp]
\centering
\includegraphics[width=\columnwidth]{figures/roc_cuve.pdf}
\caption{ROC curve for logistic regression model (AUC = 0.622).}
\label{fig:roc_curve}
\end{figure}

\subsubsection{Performance Metrics}

Based on the optimized threshold of 0.334, comprehensive performance metrics were calculated on the test set.

\textbf{Confusion Matrix:}

\begin{table}[htbp]
\centering
\caption{Confusion matrix for the logistic regression model on the test set.}
\label{tab:confusion_matrix}
\begin{tabular}{lcc}
\toprule
 & \textbf{Predicted: Alive} & \textbf{Predicted: Deceased} \\
\midrule
\textbf{Actual: Alive} & 280 (TN) & 209 (FP) \\
\textbf{Actual: Deceased} & 94 (FN) & 143 (TP) \\
\bottomrule
\end{tabular}
\end{table}

Overall accuracy was 0.583. Sensitivity and specificity were 0.603 and 0.573, respectively. The positive predictive value was 0.406, while the negative predictive value was 0.749. A summary of performance metrics is provided in Table~\ref{tab:test_performance}.

\begin{table}[htbp]
\centering
\caption{Logistic regression model performance metrics on the test set (10\% of sepsis cohort). Metrics calculated using an optimized classification threshold of 0.334.}
\label{tab:test_performance}
\begin{tabular}{lc}
\toprule
\textbf{Metric} & \textbf{Value} \\
\midrule
Accuracy & 0.583 \\
Sensitivity (True Positive Rate) & 0.604 \\
Specificity (True Negative Rate) & 0.573 \\
Positive Predictive Value (Precision) & 0.406 \\
Negative Predictive Value & 0.749 \\
\bottomrule
\end{tabular}
\end{table}





