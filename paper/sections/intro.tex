Sepsis represents a major global health burden, accounting for millions of deaths annually and substantial healthcare costs \cite{sepsis3}. Early identification and accurate epidemiological characterization are critical for improving outcomes and resource allocation in critical care settings. Administrative databases such as the Medical Information Mart for Intensive Care III (MIMIC-III) offer rich, real-world data to study sepsis incidence and outcomes at scale \cite{mimic}.

While clinical criteria like SOFA or qSOFA are increasingly used in research, ICD-9-CM diagnosis codes remain widely employed in retrospective epidemiological studies due to their availability and standardization across hospital records. However, the validity of ICD-9-based sepsis definitions has been debated, particularly regarding sensitivity and specificity \cite{icd9_sepsis_validity}.

In this study, we aim to: (1) estimate the proportion of adult ICU admissions with a sepsis-related ICD-9-CM code in MIMIC-III, and (2) compare in-hospital mortality between patients with and without such a diagnosis. As a foundational step, we first define and characterize the analytical cohort upon which all subsequent analyses are based. 



Sepsis is a life-threatening syndrome of organ dysfunction caused by a dysregulated host response to infection \cite{sepsis_definition}. It represents a critical emergency in medicine and is especially pertinent in intensive care units (ICUs), where the sickest patients are treated. Clinically, sepsis is part of a continuum of disease severity. In this context, septicemia refers to the presence of pathogens in the bloodstream, traditionally indicating bacteremia but without necessarily implying organ failure \cite{sepsis2}. Sepsis occurs when an infection triggers a systemic inflammatory response, while severe sepsis is characterized by sepsis accompanied by acute organ dysfunction. The most critical form, septic shock, involves persistent hypotension despite adequate fluid reconstruction and is associated with a very high risk of death \cite{sepsis2}. Sepsis remains a leading cause of death among ICU patients and is a major contributor to mortality and critical illness worldwide \cite{sepsis3}. In the United States alone, it accounts for a substantial healthcare burden (over 20 billion dollars in hospital costs in 2011) and its incidence has been rising in recent years \cite{sepsis3}. These statistics underscore the clinical importance of sepsis in critical care and the urgent need to better understand and address this syndrome.

Studying the epidemiology and outcomes of sepsis in ICU settings is therefore of great significance. Robust epidemiologic data can inform prevention strategies, resource allocation, and clinical decision-making in critical care. For example, understanding patient demographics, risk factors, and infection sources in sepsis is essential for designing effective prevention and early recognition programs \cite{vital_signs}. Additionally, tracking sepsis incidence and mortality over time can reveal trends and help evaluate the impact of interventions or guidelines. Notably, the reported incidence of sepsis has varied and in many regions appears to be increasing with an aging population and improved recognition \cite{sepsis3}. Given the high morbidity and mortality associated with sepsis, elucidating its epidemiology in ICUs is crucial for improving patient outcomes \cite{vital_signs}.

A challenge in retrospective research on sepsis is the identification of cases using routinely collected hospital data. In large administrative and clinical databases, sepsis is commonly identified using ICD-9-CM diagnostic codes, which reflect clinician-documented diagnoses and are widely used in epidemiological studies. Codes such as 995.91 (sepsis), 995.92 (severe sepsis), and 785.52 (septic shock) capture increasing levels of disease severity and form the basis of many retrospective analyses \cite{icd9_codes}. Alternatively, clinical scoring systems such as the SOFA score, or its simplified version qSOFA, have been proposed to identify patients with organ dysfunction and higher risk of adverse outcomes using physiological and laboratory data \cite{sofa,qsofa}. While these scores provide valuable clinical insight, their application in retrospective studies depends on the availability and completeness of detailed clinical measurements. For this reason, ICD-9-based definitions remain a practical and commonly used approach in database-driven studies such as those using MIMIC-III \cite{mimic}.

In this context, the availability of large, well-characterized critical care databases is essential for studying sepsis in a reproducible manner \cite{mimic}. The MIMIC-III database (Medical Information Mart for Intensive Care) offers a powerful resource to study sepsis in the ICU. MIMIC-III is a large, high-quality clinical database comprising de-identified health data for over forty thousand ICU patients at a tertiary academic medical center (Beth Israel Deaconess Medical Center) between 2001 and 2012 \cite{mimic}. The database includes detailed information on patient demographics, vital signs (recorded nearly hourly), laboratory results, diagnoses, procedures, medications, and outcomes, including in-hospital mortality \cite{mimic}. MIMIC-III is freely available to researchers and has been extensively used for epidemiological and outcomes research in critical care \cite{mimic}. Its large sample size and granular clinical data enable researchers to apply consistent sepsis definitions and examine outcomes with a high degree of detail and validity. In particular, MIMIC’s integration of both ICD-9 codes and physiologic data allows for flexible case definitions and validation of sepsis identification methods, making it well-suited for a retrospective analysis of sepsis epidemiology and mortality \cite{icd9_codes}.

Despite numerous studies, there remain gaps and inconsistencies in the literature regarding sepsis epidemiology in critical care \cite{epidemiology}. Historically, varying definitions of sepsis have been used, including differences in diagnostic criteria and administrative coding strategies, leading to discrepant estimates of sepsis incidence and associated outcomes across studies \cite{sepsis3}. In particular, the use of different identification approaches may result in substantial variation in the number of patients classified as septic, as well as in the severity profile of the identified populations \cite{icd9_codes}.

Consequently, reported mortality rates also vary depending on the definition applied: broader identification strategies tend to include patients with less severe illness, resulting in lower average mortality estimates, whereas more restrictive definitions capture fewer but more critically ill patients, yielding higher mortality rates \cite{icd9_codes}. These differences complicate the interpretation and comparison of mortality estimates across studies, as observed outcomes may reflect methodological choices rather than true differences in disease severity or patient populations \cite{epidemiology}. In retrospective analyses based on routinely collected hospital data, this issue is particularly relevant, as case identification depends largely on the selected diagnostic criteria and coding practices \cite{validity}. This variability highlights the need for clarity and consistency in sepsis identification when estimating prevalence and comparing outcomes in ICU populations.

In light of the above, the present study aims to characterize the epidemiology of sepsis among ICU admissions and to assess in-hospital mortality using the MIMIC-III database. Accordingly, this study addresses the question of how frequently sepsis occurs among ICU admissions and how in-hospital mortality differs between patients with and without sepsis. Specifically, we seek to (1) identify ICU patients with sepsis based on ICD-9-CM diagnostic criteria and describe their clinical characteristics, (2) estimate the prevalence of sepsis among adult ICU admissions, and (3) compare in-hospital mortality between patients with and without sepsis. In addition, among patients diagnosed with sepsis, a generalized linear model is used to explore clinical factors associated with in-hospital mortality and to predict the probability of death. Through this analysis, this study provides a consistent description of sepsis prevalence and outcomes within a well-defined ICU population.
