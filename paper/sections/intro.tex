Sepsis represents a major global health burden, accounting for millions of deaths annually and substantial healthcare costs \cite{sepsis3}. Early identification and accurate epidemiological characterization are critical for improving outcomes and resource allocation in critical care settings. Administrative databases such as the Medical Information Mart for Intensive Care III (MIMIC-III) offer rich, real-world data to study sepsis incidence and outcomes at scale \cite{mimic}.

While clinical criteria like SOFA or qSOFA are increasingly used in research, ICD-9-CM diagnosis codes remain widely employed in retrospective epidemiological studies due to their availability and standardization across hospital records. However, the validity of ICD-9-based sepsis definitions has been debated, particularly regarding sensitivity and specificity \cite{icd9_sepsis_validity}.

In this study, we aim to: (1) estimate the proportion of adult ICU admissions with a sepsis-related ICD-9-CM code in MIMIC-III, and (2) compare in-hospital mortality between patients with and without such a diagnosis. As a foundational step, we first define and characterize the analytical cohort upon which all subsequent analyses are based. 



Sepsis is a life-threatening syndrome of organ dysfunction caused by a dysregulated host response to infection \cite{sepsis_definition}. It represents a critical emergency in medicine and is especially pertinent in intensive care units (ICUs), where the sickest patients are treated. Sepsis remains a leading cause of death among ICU patients and is a major contributor to mortality and critical illness worldwide \cite{sepsis3}. In the United States alone, it accounts for a substantial healthcare burden (over $20 billion in hospital costs in 2011) and its incidence has been rising in recent years \cite{sepsis3}. These statistics underscore the clinical importance of sepsis in critical care and the urgent need to better understand and address this syndrome.

